\begin{appendices}
\chapter{Code R TP 0}
\label{Code R} 
\begin{lstlisting} 
# Importation des donnees
Wine<-read.csv("~/M1 Biostat/ADM/des bases de données/wine.csv")
# Centrer reduire les variables quatitatives
Wine_cent_red<-scale(Wine[4:32])
#Barycentre du nuage de point
Barycenter<-colMeans(Wine_cent_red)
# La distance euclidienne entre le barycentre du nuage et l'origine
Dist_Eucl<-sqrt(sum(Barycenter^2))
#Inertie
Inertie=0
for(i in 1:29)
{
Inertie = Inertie + var(Wine_cent_red[,i])
}
# Poids des Appellations
tab<-cbind(Wine[,1:3],Wine_cent_red)
Chinon<-tab[tab$Labe=="Chinon",]
Bourgeuil<-tab[tab$Label=="Bourgueuil",]
Samur<-tab[tab$Label=="Saumur",]
poid_Chinon<-nrow(Chinon)/21
poid_Bourgeuil<-nrow(Bourgeuil)/21
poid_Samur<-nrow(Samur)/21
Poids<-c(poid_Chinon,poid_Bourgeuil,poid_Samur)
Poids
# Barycentre Appellation
Barycentre_Chinon<-colMeans(Chinon[,4:32])
Barycentre_Bourgeuil<-colMeans(Bourgeuil[,4:32])
Barycentre_Samur<-colMeans(Samur[,4:32])
# Normes Euclidiennes
Normes_Eucl<-c(crossprod(Barycentre_Chinon),crossprod(Barycentre_Bourgeuil),crossprod(Barycentre_Samur))
#Inertie Inter
Inertie_inter_appelation<-crossprod(Poids,Normes_Eucl)
#R2
R2<-Inertie_inter_appelation/29
# Nos barycentres
B_Matrix <- matrix(c(Barycentre_Chinon,Barycentre_Bourgeuil,Barycentre_Samur), nrow = 29, ncol = 3)
colnames(B_Matrix) <- c("B.Chinon","B.Bourgueuil","B.Samur")
rownames(B_Matrix) <- c(1:29)
B_Matrix
l = 0
Vec_R2 <- vector(mode ="numeric", length = 29)
for (i in 1:29)
{
N = 0
for (j in 1:3)
{
N = (B_Matrix[i,j])^2*Poids[j] + N
}
l = 1 + l
Vec_R2[l] <- N
}
Matrix_R2 <- matrix(c(Vec_R2), nrow=29, ncol=1, byrow=T)
row.names(Matrix_R2)<-colnames(Wine[,4:32])
colnames(Matrix_R2)<-c("R2")
# R2 = R2
sum(Matrix_R2)/29
\end{lstlisting} 
\end{appendices}